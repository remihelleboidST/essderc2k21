\documentclass[12pt,a4paper,twocolumn]{article}
\usepackage[utf8]{inputenc}
\usepackage{authblk}
\usepackage{amsmath}
\usepackage{amsfonts}
\usepackage{amssymb}
\usepackage{makeidx}
\usepackage{graphicx}
\usepackage{fourier}
\usepackage[left=2cm,right=2cm,top=2cm,bottom=2cm]{geometry}

% Title
\title{PDE and Jitter simulation in 3D SPAD Devices.}

% AUthors
\author[1]{Rémi Helleboid}
\author[1]{Denis Rideau}
\affil[1]{ST Microelectronics, Crolles, France}

\date{}                     %% To hide the date
\setcounter{Maxaffil}{0}
\renewcommand\Affilfont{\itshape\small}
% Keywords command
\providecommand{\keywords}[1]
{
  \small	
  \textbf{\textit{Keywords---}} #1
}


\begin{document}


% Title
\maketitle

% Abstract
\begin{abstract}
In this paper we present a full 3D simulation methodology to extract Photon Detection Probability (PDP) and Jitter of Single-Photon Avalanche Diode (SPAD) Devices. The simulation results are compared with measurements on devices and show good agreement with the experiments.\\
\end{abstract}

% Keywords
\keywords{single-photon avalanche diode (SPAD), photon detection probability (PDP), jitter, avalanche breakdown probability, breakdown voltage}

% Introduction
\section{Introduction}
A single-photon avalanche diode (SPAD) is a solid-state photodetector within the same family as photodiodes and avalanche photodiodes (APDs), while also being fundamentally linked with basic diode behaviours. As with photodiodes and APDs, a SPAD is based around a semi-conductor p-n junction that can be illuminated with ionizing radiation such as gamma, x-rays, beta and alpha particles along with a wide portion of the electromagnetic spectrum from ultraviolet (UV) through the visible wavelengths and into the infrared (IR). 





\end{document}